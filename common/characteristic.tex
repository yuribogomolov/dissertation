{\actuality}
Многие процессы и события в современном мире могут быть представлены в виде транзакционных пространственно-временных рядов. Примеры включают общение по телефону, поездки на транспорте, операции покупки/продажи, взаимодействие с социальными сетями. Каждое из этих событий представляет собой транзакцию, которая совершается в пространстве и времени. 

Развитие мобильных технологий и информационных систем привело к экспоненциальному росту объема пространственно-временных данных, что в свою очередь увеличило привлекательность исследований в этом направлении \autocite{roddick1999bibliography,cressie2015statistics,diggle2013statistical}. 

Большой интерес для урбанистики представляют собой транзакционные данные, которые описывают взаимодействие между парами или наборами объектов \cite{hawelka2014geo,paldino2015urban, ratti2010redrawing, sobolevsky2013delineating}. Транзакции могут представлять разговоры по мобильному телефону \cite{belyi2017global, kung2014exploring, amini2014impact}, операции с кредитными картами \cite{sobolevsky2016cities}, поездки на такси \cite{santi2014quantifying} и многие другие события реального мира. Транзакции помогают описать  и проанализировать социальные и экономические взаимодействия между людьми, либо между человеком и юридическим лицом.

{\progress}
Сложный формат входных данных и наличие временных, пространственных и транзакционных составляющих усложняют процесс исследования и моделирования. Даже создание простых моделей требует большого объема подготовительной работы. Существенная часть данной проблемы объясняется тем, что используемые библиотеки не предоставляют необходимые абстракции и функции для работы с пространственно-временными транзакциями.

{\aim} данной работы является разработка новых методов моделирования социального взаимодействия жителей города в пространстве и времени.

Для~достижения поставленной цели необходимо было решить следующие {\tasks}:
\begin{enumerate}[beginpenalty=10000] % https://tex.stackexchange.com/a/476052/104425
  \item Проанализировать стандартные методы моделирования взаимодействия.
  \item Разработать модель данных, которая обобщает сетевые, пространственные и временные данные внутри одной модели.
  \item Исследовать зависимость мобильности и социального взаимодействия от экономических и социальных факторов.
  \item Предложить новые методы моделирования и поиска аномалий на основе обобщенной модели пространственно-временных транзакционных данных.
\end{enumerate}


{\novelty}
\begin{enumerate}[beginpenalty=10000] 
  \item Впервые было предложено рассматривать пространственно-временные сетевые транзакции как новый класс данных.
  \item Впервые популярные методы моделирования социального взаимодействия были обобщены на случай произвольных пространственно-временных данных. 
  \item Было выполнено оригинальное исследование на тему зависимости мобильности и социального взаимодействия от экономических и социальных факторов.
\end{enumerate}


{\influence} работы состоит в разработке программной библиотеки для анализа пространственно-временных данных. Исходный код библиотеки выложен в открытый доступ, что позволяет переиспользовать реализованные методы фильтрации, агрегации и анализа любой сети. 

{\methods}
Классические методы моделирования взаимодействия в пространстве: модели гравитации и радиации.


{\reliability} полученных результатов обеспечивается предоставлением открытого исходного кода и возможностью воспроизведения полученных результатов любым заинтересованным лицом.

{\probation}
Основные результаты работы докладывались~на:
\begin{enumerate}[beginpenalty=10000]
  \item X Конгресс молодых ученых
  \item 10th International Young Scientists Conference in Computational Science (на рассмотрении)
\end{enumerate}

{\contribution} Автор играл ведущую роль в решении всех указанных задач.
