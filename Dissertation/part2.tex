\chapter{Пространственно-временные сети}\label{ch:ch1}

\section{Обзор существующих решений}\label{sec:ch1/sec1}

\subsection*{Pandas}
\textbf{Pandas} - самая популярная библиотека для работы с табличными данными.  Данное решение широко используется для анализа временных рядов. Библиотека предоставляет возможность задавать структуру данных и индексы.

Pandas хорошо справляется с временной составляющей анализа данных, но библиотека не содержит абстракций для пространственных и транзакционных компонент.

\subsection*{Shapely}
\textbf{Shapely} - библиотека для пространственного анализа и манипуляций 2-D объектов на основе GEOS. Shapely полностью сфокусирована на богатом геометрическом функционале, и библиотека хороша справляется с этой задачей.

\subsection*{GeoPandas}
\textbf{{GeoPandas}} - попытка объединить функционал двух предыдущих библиотек в одно целое. GeoPandas предоставляет новый уровень абстракции с помощью объектов GeoDataFrame и GeoSeries. Данный функционал библиотеки удобен для работы с набором 2-D объектов, и их модификациями во времени. Представленные абстракции не могут быть использованы напрямую для анализа взаимодействий субъектов в пространстве и времени. 

\subsection*{PySAL}
\textbf{PySAL} (Python Spatial Analysis Library)  - библиотека основанная на GeoPandas для анализа и визуализации пространственных данных. PySAL может строить пространственные регрессии, линейные и нелинейные пространственные статистические модели. Библиотека также содержит ограниченный функционал для работы с пространственно-временными рядами.
Отсутствие транзакционной составляющей в модели данных и основных объектах системы существенно усложняет ее использование для решения поставленных выше задач. 
 
\subsection*{IBM PAIRS}
\textbf{IBM PAIRS} - облачное решение от компании IBM для работы с пространственно-временными данными. Данный продукт предоставляет легкий способ получать нужные данные из облака на основании временных и пространственных фильтров.
PAIRS рассчитан на простые манипуляции с большими данными, на текущий момент (Январь 2021) продукт имеет только функционал поиска, фильтрации и агрегирования.

\subsection*{Geomesa}
\textbf{Geomesa} - система для быстрой аналитики пространственно-временных данных большого объема. Основная идея Geomesa - это представление возможности индексации геометрических объектов (точка, отрезок, прямоугольник) с помощью стандартных распределенных хранилищ данных (HBase, Google Bigtable, Cassandra).
Аналогично с IBM PAIRS, данное решение ориентировано на простые операции с большими объемами данных.

\subsection*{BigQuery GIS}
\textbf{BigQuery GIS} - облачное решение от компании Google для анализа и визуализации больших пространственных данных. BigQuery GIS является частью платформы BigQuery, которая ориентирована на эффективную обработку больших объемов данных. Данное решение поддерживает только стандартные геометрические функции над объектами.


\subsection*{Вывод}
Рассмотренные библиотеки решают обобщенные задачи временного, пространственного и пространственно-временного анализа. Ни одна из рассмотренных моделей данных не поддерживает транзакционность данных. И как следствие не может быть напрямую применена для анализа взаимодействия объектов разных видов в пространстве и времени. 


\section{Обобщенное представление пространственно-временной сети}\label{sec:ch1/sec2}

Стандартный анализ пространственно-временных транзакционных данных имеет следующую структуру:
\begin{enumerate}
  \item Выделение объектов и транзакций (которые представляют связи между объектами)
  \item Построение пространственно-временной сети в которой исследуемые объекты представлены вершинами, а взаимодействие между объектами (транзакции) представлены в виде дуг
  \item Фильтрация
  \item Валидация данных (включая временное и пространственное покрытие)
  \item Агрегация сети 
  \item Последующий анализ сети (в зависимости от области исследования)
\end{enumerate}

Этапы 3-5 опираются на пространственное и временное измерение транзакций, а этапы 1, 2 и 5 фокусируются на сетевом представлении данных. Последующий анализ (Этап 6) может включать как пространственно-временные так и сетевые компоненты. Многогранность объекта исследования существенно усложняет обработку данных и требует многочисленных переходов от временного и пространственного представления к сетевой модели и обратно. 

В данной работе мы предлагаем обобщенное представление пространственно-временной сети, которое позволяет производить временную, пространственную и сетевую обработку данных внутри одной модели.

Наше представление основано на ориентированном псевдографе с метками для вершин и ребер. Более формально: 
\begin{equation}
\label{eq:sttn} {
STTN = (V, A, \Sigma_V, \Sigma_A, o, d, L_V, L_A, space, time)
}
\end{equation}

V - множество вершин

А - множество дуг

$\Sigma_V$ - алфавит меток множества вершин

$\Sigma_A$ - алфавит меток множества дуг

$o: A \mapsto V$ - функция соответствия начальной вершины дуги

$d: A \mapsto V$ - функция соответствия конечной вершины дуги

$L_V: V \mapsto \Sigma_V$ - функция меток для вершин

$L_A: A \mapsto \Sigma_A$ - функция меток для дуг

$space$ - пространственная характеристика $\Sigma_V$

$time$ - временная характеристика $\Sigma_A$


Следует отметить что алфавиты $\Sigma_V$ и $\Sigma_A$ могут содержать дополнительные пространственные и временные компоненты. Функции $space, time$ отвечают за выбор пространственного и временного измерения, которое  используется моделью для пространственного/временного анализа.

\section*{Операции преобразования пространственно-временной сети}

Используя указанное представление мы можем определить следующие операции:

\subsection*{Агрегация параллельных дуг}

Параллельными называются дуги ориентированного у которых совпадают концевые вершины. В нашей модели представления ПВТД параллельные дуги могут представлять поездки на такси между задаными районами города, или телефонные звонки между двумя фиксированными городами. В процессе агрегации для каждого набора параллельных дуг нам необходимо решить следующие задачи:

\begin{itemize}
  \item Определить соответствие между дугами исходной сети и дугами агрегированной сети
  \item Вычислить метки для ребер агрегированной сети на основании меток исходной сети
\end{itemize}

Таким образом мы получаем следующую операцию:

\textbf{agg\_parallel\_arcs(key, arc\_aggregation)} , где:

$key: A \mapsto A_{aggregated}$ - функция соответствия между дyгами исходной и агрегированной сети. В самом простом случае каждый набор параллельных ребер исходной сети преобразуются в одно единственное ребро (например нас может интересовать средняя продолжительность всех поездок между вершинами $X$ и $Y$). Однако в более общем случае  может понадобиться более гибкий пример агрегации. Например вычисления средней стоимости поездки на такси между заданными районами в зависимости от времени суток.  

$arc\_aggregation: list(L_A) \mapsto L'_A$

\textbf{group\_nodes} - комбинирование вершин

\textbf{agg\_adjacent\_edges} - агрегация инцидентных дуг

\textbf{join\_node\_labels} - расширение алфавита меток вершин

\textbf{filter\_nodes} - фильтрация вершин и инцидентных им дуг

